\documentclass[10pt,a4paper]{report}
%\usepackage[latin1]{inputenc}
\usepackage{blindtext}
\usepackage{multicol}
\usepackage[utf8]{inputenc}
\usepackage{amsmath}
\makeatletter
\newcommand\xleftrightarrow[2][]{%
  \ext@arrow 9999{\longleftrightarrowfill@}{#1}{#2}}
\newcommand\longleftrightarrowfill@{%
  \arrowfill@\leftarrow\relbar\rightarrow}
\makeatother
\usepackage{amsfonts}
\usepackage{amssymb}
\usepackage{graphicx}
\usepackage{multicol}
\usepackage{tabularx}
\usepackage{tikz}
\usetikzlibrary{arrows,shapes,automata,petri,positioning,calc}
\usepackage{hyperref}
\usepackage{tikz}
\usetikzlibrary{matrix,calc}
\usepackage[margin=0.5in]{geometry}
\providecommand{\norm}[1]{\left\lVert#1\right\rVert}
\newcommand{\myvec}[1]{\ensuremath{\begin{pmatrix}#1\end{pmatrix}}}
\let\vec\mathbf

\newcommand{\mydet}[1]{\ensuremath{\begin{vmatrix}#1\end{vmatrix}}}

%\newcommand{\myvec}[1]{\ensuremath{\begin{pmatrix}#1\end{pmatrix}}}
%\let\vec\mathbf
\providecommand{\mtx}[1]{\mathbf{#1}}
\newenvironment{Figure}
  {\par\medskip\noindent\minipage{\linewidth}}
  {\endminipage\par\medskip}
\begin{document}
%--------------------logo figure-------------------------%
\begin{figure*}[!tbp]
  \centering
  \begin{minipage}[b]{0.4\textwidth}
   \includegraphics[scale=0.5]{iithlogo.png} 
  \end{minipage}
  \hfill
  \vspace{5mm}\begin{minipage}[b]{0.4\textwidth}
\raggedleft \includegraphics[scale=0.5]{nrc.jpeg} 

  \end{minipage}\vspace{0.2cm}
\end{figure*}
%--------------------name & rollno-----------------------
\raggedright 

\begin{center}
\Large \textbf{Line Assignment}\hspace{2.5cm} %
\end{center}
\begin{center}
\textbf{Name}:\hspace{1mm}Rupa Sai Sreshta Vallabhaneni\hspace{3cm}
\end{center}
  
%\normalsize \textbf{Roll No.} :\hspace{1mm} FWC22047\vspace{1cm}
%\begin{multicols}{2}

%\begin{tableofcontents}
%\begin{tableofcontents}
\begin{multicols}{2}
\section{Problem}

\noindent The area of triangle is 5. Two of its vertices are A(2,1) and B(3,-2). The third vertex lies on y=x+3. Find C.\\
\includegraphics[scale=0.4]{tri.png}\\
%\includegraphics[scale=0.4]{construction.png}
%\begin{center}
Figure of Construction\\
%\end{center}
%\end{abstract}
The following python code is used for finding third vertex of triangle 
		  \\
		 \boxed{Github link: \href{https://github.com/RupaSaiSreshta/FWC}{https://github.com/RupaSaiSreshta/FWC}}
\section{Solution}
%\justify
\textbf{Construction: }\\
%Given vertices 
%A(2,1) B(3,-2)
%\\
%Area of triangle: 5
%\\
%Third vertex C lies on y=x+3 

\textbf{Input parameters for this construction}
\begin{center}
\begin{tabular}{|c|c|}
\hline
\textbf{Symbol}&{Value}\\
\hline
A&(2,1)\\
\hline
B&(3,-2)\\
\hline
Area&5\\
\hline
C lies on&y=x+3\\
\hline
\end{tabular}
\end{center}
\textbf{  Solution:}\\
Let us assume C(x,y)\\
Area of triangle is \\
%\centering
 \textbf
{ $$ Ar=\frac{1}{2}\left| \vec{AB}X\vec{AC} \right| $$}
   %Now,\\
   %A=1/2[x1(y2-y3)+x2(y3-y1)+x3(y1-y2)]......(1)\\
   %Here,\\
   %A=5 ; x1=2 ; x2=3 ; x3=x ; y1=1 ; y2=-2 ; y3=y\\
    %Substitute these values in above equation (1)\\
    %5=1/2[2(-2-y)+3(y-1)+x(1+2)]
\\   % 10=3x+y-7\\
By solving we get\\
  3x+y=17......(1)\\
  The vertex C lies on y=x+3......(2)  \\	 
		    %\begin{align}
%			    y &= x+3
%			    \\
%			    x+3 &= 2\brak{y+3} + 10
		    %\end{align}
		   % which can be expressed as 
		  
		   By solving (1) and (2) equations using matrix reduction method\\
		    \begin{align}
			    3x  +y &= 17
			    \\
			    x -y  &= -3
			    \\
			    %The above equations are in the form of Ax=B\\
			   % When we write in matrix form
			    \implies 
			    \myvec{3 &  1 
			    \\
			    1 & -1 }\vec{x} &= \myvec{17 \\ -3}
		    \end{align}
		    The above matrix is in the form of Ax=B\\
		    The augmented matrix for the above matrix equation is 
		    \begin{align}
			    \myvec{
				    3 & 1 & \vrule & 17
			    \\
			    1 & -1  &\vrule & -3
		    }
		    \\
		    \xleftrightarrow[]{R_2 \leftarrow 3R_2 -R_1 }
			    \myvec{
				    3 & 1 & \vrule & 17
			    \\
			    0 & -4  &\vrule & -20
		    }
		    \\
		    \xleftrightarrow[]{R_1 \leftarrow 4R_1 +R_2 }  
			    \myvec{
				    12 & 0 & \vrule & 42
			    \\
			    0 & -4  &\vrule & -26
		    }
		    \\
		     \xleftrightarrow[]{R_1 \leftarrow R_1* 1 /12 }
			    \myvec{
				    1 & 0 & \vrule & 3.5
			    \\
			    0 & -4  &\vrule & -20
			    }
			    \\
			     \xleftrightarrow[]{R_2 \leftarrow R_2* 1 /4 }
			    \myvec{
				    1 & 0 & \vrule & 3.5
			    \\
			    0 & 1  &\vrule & 6.5
			    }
			    \implies \vec{x} = \myvec{3.5 \\ 6.5}
\end{align}
		    \\
		    Now the vertex of C is
		    \begin{center}
		    \boxed{ C(3.5,6.5)}
		    \end{center}
		    
		  
		   
 \section{Proof}
 \textbf{Proof:}
Here The area of triangle is given as 5.Now we have to calculate the area of triangle using three vertices A,B and C. We need to prove that the area of triangle is 5.
\\Area of triangle is
{ $$ Ar=\frac{1}{2}\left| \vec{AB}X\vec{AC} \right| $$}
\\By substituting the values of A,B and C\\
{ $$ Ar=\frac{1}{2}[2(-8.5)+3(5.5)+3.5(3)] $$ }
{ $$ Ar=\frac{1}{2}[-17+16.5+10.5] $$ }
{ $$ Ar=\frac{1}{2}[10.5-0.5] $$ }
\centering
\boxed {Ar=5}
\\Hence verified.
%\end{tableofcontents}
\end{multicols}
%\end{tableofcontents}
\end{document}
