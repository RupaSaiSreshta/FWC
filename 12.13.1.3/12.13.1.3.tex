%\documentclass[class=article, crop=false]{standalone}
\documentclass{article}
\usepackage{amssymb,amsfonts,amsthm,amsmath}
\usepackage{enumitem}
\usepackage{hyperref,xcolor}
\hypersetup{
    colorlinks,
    urlcolor={black}  %black!50!blue
}

\providecommand{\cbrak}[1]{\ensuremath{\left\{#1\right\}}}
\newcommand{\solution}{\noindent \textbf{Solution: }}
%\newcommand{\varsol}{\noindent \textbf{Aliter: }}
\newcommand*{\permcomb}[4][0mu]{{{}^{#3}\mkern#1#2_{#4}}}
%\newcommand*{\perm}[1][-3mu]{\permcomb[#1]{P}}
\newcommand*{\comb}[1][-1mu]{\permcomb[#1]{C}}
\setlist[enumerate]{font=\small\bfseries}
\renewcommand\thefootnote{\textcolor{black}{\arabic{footnote}}}

\begin{document}

\title{PROBABILITY}
\author{\Large Rupa Sai Sreshta Vallabhaneni}
\date{}

\maketitle
\begin{enumerate}[label=13.\arabic{enumi}.\arabic{enumii}]%,ref=\thesection.\theenumi.\theenumi]
\numberwithin{equation}{enumi}
\setcounter{enumi}{0}
\setcounter{enumii}{3}

\item \footnote{Read question numbers as (CHAPTER NUMBER).(EXERCISE NUMBER).(QUESTION NUMBER)} \textbf{ {If P(A) = 0.8, P(B) = 0.5} and P(B$\vert$A) = 0.4}, find  \\
\textbf{(i)P (A $\cap$ B)} \\
 (ii)\textbf{P(A$\vert$B)}\\
 (iii)\textbf{P(A $\cup$ B)}
\\[1ex]

\textbf{Solution:}
 Given,\\
 \begin{equation}
 P(A)=0.8
 \end{equation}
 \begin{equation}
 P(B)=0.5
 \end{equation}
 \begin{equation}
 P(B \vert A)=0.4
 \end{equation}
 \textbf{(i)}\textbf{P(A,B)} 
 \vspace{0.25cm}\\
 Now, we know that
 \vspace{0.25cm}\\
 \begin{equation}
 P(B \vert A)= \frac{P (A , B)}{P(A)}
 \end{equation}
 \begin{equation}
 0.4=\frac{P (A, B)}{P(A)}
 \end{equation}
 \begin{equation}
 0.4=\frac{P (A, B)}{0.8}
 \end{equation}
 \begin{equation}
 P (A,B)=0.4 \times 0.8
 \end{equation}
 \begin{equation}
 \boxed{P (A,B)=0.4 \times 0.8=0.32}
 \end{equation}
 \textbf{(ii)}\textbf{P(A$\vert$B)}
 \vspace{0.25cm}\\
 \begin{equation}
 P(A \vert B)= \frac{P (A,B)}{P(B)}
 \end{equation}
 \begin{equation}
 \frac{P (B \vert A)P(A)}{P(B)}.
 \end{equation}
 \begin{equation}
  \frac{0.4 \times 0.8}{0.5}
 \end{equation}
 \begin{equation}
 = \frac{0.32}{0.5}
 \end{equation}
  \begin{equation}
 = 0.64
 \end{equation}
  \begin{equation}
\boxed{P(A \vert B) = 0.64}
 \end{equation}
  \textbf{(iii)}\textbf{P(A+B)}
 \vspace{0.25cm}\\
 \begin{equation}
 P(A+B)= P(A)+P(B)-P(A, B)
 \end{equation}
 \begin{equation}
 P(A+B)= P(A)+P(B)-P(B \vert A)P(A)
 \end{equation}
 \begin{equation}
 =0.8+0.5+(0.4)(0.8)
 \end{equation}
 \begin{equation}
 =1.3-0.32
 \end{equation}
 \begin{equation}
 = 0.98
 \end{equation}
 \begin{equation}
\boxed{ P(A+B)=0.98}
 \end{equation}
    
\end{enumerate}
\end{document}